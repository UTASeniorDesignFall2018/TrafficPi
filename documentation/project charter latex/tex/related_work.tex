The current state of the traffic study arena is mainly focused on using existing traffic cameras and performing traffic analysis on it. For example, Logipix Technical Development Ltd. is a company that specializes in high-end video surveillance solutions (Logipix, 2017). This company uses professional grade cameras to monitor airports, highways, stadiums, or big areas of interest. Another company TrafficVision specializes in analyzing prerecorded footage for customers and provided specific data reports for them (TrafficVision, 2018). 

Aside from the above two, there are currently two companies that have similar products to what this team is trying to achieve. 

The first is Miovision Technologies Inc. This company speciliazes in a broad area for performing traffic studies. They are able to capture video of intersections, park trails, neighborhoods, crosswalks, and more. The data is analyzed and presented to the customer in a concise usable format for making traffic decisions (Miovision, 2018). The company can use the existing cameras placed by the city or deploy its own equipment on site. 

The second company is Roadometry. This company is similar to Miovision in that it can deploy its own camera systems, record, traffic, and analyze it as well (Roadometry, 2018).

The difference between the products and solutions these companies offer and the direction this team is taking with the Traffic Pi project is that, their solutions are marketed to city customers with correpsonding price tags. The Traffic Pi aims to market to individual homeowners or neighborhoods.