The current method by which a traffic study takes place can be generalized by the following. A need for such a study arises in a neighborhood (multiple accidents over a period of time at a particular intersection, multiple close-calls at a particular intersection, etc.). Either the city recognizes a potential problem in the area of interest or a resident in the community makes known to the city the area and requests something be done. The city will usually follow up on the request with a traffic study; monitoring the area with pneumatic road tubes, wires laid on the ground that monitor the speed a of vehicle as it passes over it. Depending on the area of interest, multiple road tubes may be used along with other equipment or actual persons on the ground monitoring the area. After the area has been monitored for a given period of time a decision is made on the area (install stop sign, signal lights, road bump, etc.). There are several problems with the current model.

To begin, the ultimate decision to initiate a traffic study begins with the city. A citizen, neighborhood, or even business park may request/petition the city to investigate potentially dangerous area but these are only requests. If in fiscally troubling times the funds for such a study are lacking, the study may not even be feasible and thus not occur. Additionally if the city for whatever reason (politically motivated, not enough complaints, etc.) decides not to investigate the area, the public is not able to have their request satisfied. 

If a city does decide to begin a traffic study, the cost of such a study may limit the scope or duration of the study. On average the cost of a single pneumatic road tube can range from \$500 to the \$1000's. This cost does not include the installation cost, environment hazards, maintenance, and otherwise. This cost is heavy on both large and small cities. In large cities, there is likely an abundance of areas that require a traffic study and thus multiple devices must be deployed. In small cities the budget for such a device may not exist at all. In short, the cost of the current method is high.

Lastly, in relating to the above two points, the ability to perform the study lies solely within the city because of power and money. The individual citizen, neighborhood, or entity is not able due to the reasons above able to conduct this study on their own. Relegating what should be a local affair to city governance. 