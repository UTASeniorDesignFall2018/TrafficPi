The customer requirements for Traffic Pi are those that were deemed to be the core essentials of this project by the team stakeholder, Dr. McMurrough. Each of the customer requirements described below define a function of the Traffic Pi system either in use, post-use, or as a commercial product. Each of the requirements defined below are critical to the Traffic Pi and can be used as a benchmark for determining the success of this system.

\subsection{Find Vehicle}
\subsubsection{Description}
The program should be able to find a vehicle inside of the frame. This means that if there are multiple cars in the frame at the same time, the second car's position and speed should not affect the position and speed of the first car. 
\subsubsection{Source}
Customer
\subsubsection{Constraints}
One assumption that may need to be made is that the entire car can fit into the frame. If it is too big (camera too close) it could negatively impact the calculation of our algorithm. 
\subsubsection{Standards}
We will use YOLOv3 as our image recognition software, so we will follow any standards they recommend using.
\subsubsection{Priority}
Critical

\subsection{Track Vehicle}
\subsubsection{Description}
The program should be able to track a vehicle inside of the frame. This means that if there are multiple cars in the frame at the same time, the second car's position and speed should not affect the position and speed of the first car. 
\subsubsection{Source}
Customer
\subsubsection{Constraints}
One assumption that may need to be made is that the entire car can fit into the frame. If it is too big (camera too close) it could negatively impact the calculation of our algorithm. 
\subsubsection{Standards}
We will use YOLOv3 as our image recognition software, so we will follow any standards they recommend using.
\subsubsection{Priority}
Critical

\subsection{Calculate Speed of Vehicle}
\subsubsection{Description}
The program will be able to calculate the speed the car is moving by determining the distance between two points and using a pixel to distance algorithm for each frame. 
\subsubsection{Source}
Customer
\subsubsection{Constraints}
Assume the ability to constantly detect the vehicle from when they enter the frame to when they exit the frame.
\subsubsection{Standards}
N/A
\subsubsection{Priority}
Critical

\subsection{Interface}
\subsubsection{Description}
Traffic Pi will provide a simple interface (web or otherwise) to display the data collected. 
\subsubsection{Source}
Customer
\subsubsection{Constraints}
N/A
\subsubsection{Standards}
N/A
\subsubsection{Priority}
Critical

\subsection{Cost}
\subsubsection{Description}
Traffic Pi will cost a reasonable amount to develop.
\subsubsection{Source}
Customer
\subsubsection{Constraints}
This strongly depends on the current cost of materials on the market
\subsubsection{Standards}
N/A
\subsubsection{Priority}
Moderate

\subsection{Affordability}
\subsubsection{Description}
Traffic Pi to be affordable to the average consumer
\subsubsection{Source}
Customer
\subsubsection{Constraints}
N/A
\subsubsection{Standards}
N/A
\subsubsection{Priority}
Moderate

\subsection{Mounting}
\subsubsection{Description}
The housing case for the Traffic Pi shall be designed such that it allows a user to connect a tripod or other mounted equipment to the Traffic Pi.
\subsubsection{Source}
Customer
\subsubsection{Constraints}
N/A
\subsubsection{Standards}
The mounting area has an area that fits to standard mounting plate designs
\subsubsection{Priority}
Critical

\subsection{Marketability}
\subsubsection{Description}
The Traffic Pi will be unique in its design and appealing to the public.
\subsubsection{Source}
Customer
\subsubsection{Constraints}
N/A
\subsubsection{Standards}
N/a
\subsubsection{Priority}
Critical

\subsection{Ease of use}
\subsubsection{Description}
Goal is to minimize the user interaction with the system
\subsubsection{Source}
Customer
\subsubsection{Constraints}
N/a
\subsubsection{Standards}
N/a
\subsubsection{Priority}
High