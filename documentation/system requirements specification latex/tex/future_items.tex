The future of Traffic Pi is a cloud connected network of devices deployed across a city. Each deployed system would be able to network with a cloud server that in turn could network with other Traffic Pi systems across the city. Together the combined system's would be able to given a city-wide analysis of traffic. Further, the processing for such a system would occur on the cloud server.

\subsection{Compatibility With Cloud System}
\subsubsection{Description}
Optimally, a future version of the product should be able to record its data and then send the long video files to a system on the cloud.  This system will then run the programs that will actually analyze and record the traffic data from the Raspberry Pi.  This means that the Pi should be able to accurately and efficiently communicate without sources of different Operating Systems.
\subsubsection{Source}
Ethan Duff
\subsubsection{Constraints}
There are no relevant constraints for this requirement, though that is subject to change.
\subsubsection{Standards}
N/A
\subsubsection{Priority}
Low


\subsection{Pi Network}
\subsubsection{Description}
Should the product become a success a possible direction to take the project would be into networked Pi's.  Multiple Pi's would be set across the city and monitor many relevant intersections to gather a much wider and more diverse data set relevant to many different portions of a city at many different times.
\subsubsection{Source}
Ethan Duff
\subsubsection{Constraints}
Hardware constraints only.
\subsubsection{Standards}
Only the legal standards enforced by any given city.
\subsubsection{Priority}
Low